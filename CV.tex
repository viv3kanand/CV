
\documentclass[a4paper, 10pt]{article}
\usepackage[margin=3cm]{geometry}
\usepackage{helvet}

\usepackage{hyperref}
\hypersetup{
   colorlinks,
   citecolor=black,
   filecolor=black,
   linkcolor=black,
   urlcolor=blue
}

\usepackage{url}
\usepackage{bibentry}

% Reformat the date
% http://www.howtotex.com/packages/customize-the-date-format-in-your-latex-documents/
% \usepackage{datetime}
% \newdateformat{mydate}{\THEYEAR\ \monthname[\THEMONTH] \THEDAY\textsuperscript{th}}

% \author{
%    \href{mailto:viv3kanand@gmail.com}{viv3kanand@gmail.com}
% }
% \date{\mydate\today}

% remove date and whitespace from title
\date{\vspace{-10ex}}

\usepackage{array, xcolor}
\definecolor{lightgray}{gray}{0.8}
\newcolumntype{L}{>{\raggedleft}p{0.14\textwidth}}
\newcolumntype{R}{p{0.8\textwidth}}
\newcommand\VRule{\color{lightgray}\vrule width 0.5pt}

\begin{document}
% remove space above title
\title{\vspace{-12ex}Vivekanand A}

\maketitle

\parbox[t]{0.5\textwidth}{
\begin{tabbing}
\hspace{3cm} \= \hspace{4cm} \= \kill
{\bf Address} \> Thulasitharayil, Thattamala P.O,\\
\> Kollam, Kerala, India. 691020 \\
{\bf Date of Birth} \> 1990 June 25\textsuperscript{st} \\
{\bf Nationality} \> Indian \\
\end{tabbing}
}
\hfil
\parbox[t]{0.5\textwidth}{
\begin{tabbing}
\hspace{2cm} \= \hspace{4cm} \= \kill
{\bf GitHub} \> \href{https://github.com/viv3kanand}{https://github.com/viv3kanand} \\
%{\bf Twitter} \> \href{https://twitter.com/viv3kanand}{https://twitter.com/viv3kanand} \\
{\bf Email} \> \href{mailto:viv3kanand@gmail.com}{viv3kanand@gmail.com} \\
{\bf Phone} \> +91 9496819723
\end{tabbing}
}

%\section*{Summary}

%I am currently a post-doctoral researcher at the University of Western Australian working on single cell transcriptomics. Prior to this position, I was working on the analysis of whole exome sequencing in patients with rare genetic disorders at the Telethon Kids Institute. During my PhD I was a Marie Curie Early Stage Researcher in the lab of Piero Carninci in RIKEN Yokohama and was working primarily on the analysis of high-throughput transcriptome sequencing data sets. I have also worked as a bioinformatician at the University of Queensland and the Commonwealth Scientific and Industrial Research Organisation.

\section*{Education}
\begin{tabular}{L!{\VRule}R}
   2010--2015 & MSc. Bioinformatics, Amrita VishwaVidyapeetham, Amritapuri Campus, Kerala. Percentage of Marks: 66 \\
   2001--2005 & BSc. Microbiology, . Amrita VishwaVidyapeetham, Amritapuri Campus, Kerala. Percentage of Marks: 60 \\
\end{tabular}

\section*{Past Scientific Positions}
\begin{tabular}{L!{\VRule}R}
   2017--2018 & Research Fellow at Indian Institute of Technology, Delhi, India \\
   2017--2017 & Research Assistant at Indian Institute of Science, Bengaluru, India \\
   2014--2017 & Research Fellow at Rajiv Gandhi Centre for Biotechnology, India \\
\end{tabular}

 \section*{Research Interests}
 
 \begin{itemize}
    \setlength\itemsep{0em}
    \item Investigating the potential role of genetic variants in relation to biological function and disease.
    \item Genomics and transcriptomics; in particular the study of Super-enhancers, non-coding RNAs, RNA-binding proteins and transposable elements.
    \item The application of bioinformatics using Next Generation Sequencing and statistical data analysis methods to answer biological problems.
 \end{itemize}

\section*{Honours and Awards}
\begin{tabular}{L!{\VRule}R}
   2016 & Selected for Sakura Science Program (Japan Asia Youth Exchange Program in Science) hosted by Japan Science and Technology (JST) at Okayama University, Japan\\
   2015 & Awarded International Travel Fellowship from Department of Biotechnology, India to attend a conference, RECOMB-15 held at Warsaw, Poland \\
   2014 & Organizing member in NGS- 15, Next Generation Sequencing Data Analysis Workshop hosted by EMBL-EBI and Rajiv Gandhi Centre for Biotechnology, Trivandrum, Kerala, India \\
\end{tabular}

\section*{Mentoring and Teaching experience}
\begin{tabular}{L!{\VRule}R}
   2017--2018 & Experience as Teaching Assistant for B-Tech/M-Tech/Ph.D. students at IIT Delhi \\
   2014--2017 & Teaching faculty for Bioinformatics Internship Program at Rajiv Gandhi Centre for Biotechnology \\
   2014--2017 & Teaching faculty for Bioinformatics coursework for Ph.D. students at Rajiv Gandhi Centre for Biotechnology \\
   2014--2017 & Training faculty in Next Generation Sequencing Analysis and Perl-Python workshop series organized by Rajiv Gandhi Centre for Biotechnology \\
\end{tabular}

\section*{Seminars, Workshops and Conference Attendances}
\begin{tabular}{L!{\VRule}R}
   2016 & Cancer Proteogenomics Workshop hosted by Broad Institute of MIT and Harvard and Rajiv Gandhi Centre for Biotechnology, held at Trivandrum, Kerala, India \\
   2015 & Vivekanand A, Rajesh Raju, Sreenivas KP, Reshmi G, Pillai R. "Regulatory Map of Deptor" at IACR (Indian Association of Cancer Research), held at Jaipur, Rajasthan, India \\
   2015 & Vivekanand A, Reshmi G and Pillai MR. "miRBP: An architecture to find competitive interaction between microRNA and RNA binding proteins" at RECOMB- 15, 19thAnnual International Conference on Research in Computational Molecular Biology, held at Warsaw, Poland \\
   2014 & NGS- 15, Next Generation Sequencing Data Analysis Workshop hosted by EMBL-EBI and Rajiv Gandhi Centre for Biotechnology, Trivandrum, Kerala, India \\
   2012 & NANOBIO 2012: Second International Conference on Nanotechnology, Amrita Institute of Medical Sciences, Cochin, Kerala, India \\
   2011 & Proteomics and Genomics workshop organized by Institute of Bioinformatics, Bangalore and Amrita VishwaVidyapeetham, Amrita School of Biotechnology, Amritapuri, Kerala, India \\
\end{tabular}


\section*{Bioinformatic Skills}

\begin{itemize}
   \setlength\itemsep{0em}
   \item Data analysis of high-throughput sequencing data including DNA-seq, RNA-seq, CAGE-seq, ChIP-seq, Microarry and Array CGH.
   \item Knowledge and the ability to use various bioinformatic databases, APIs, repositories, and tools.
   \item Ability to implement bioinformatic pipelines using pipelining tools.
\end{itemize}

\section*{Computer Skills}

\begin{itemize}
   \setlength\itemsep{0em}
   \item Operating systems: Linux/Unix (RHEL/CentOS and Ubuntu), OS X, and Windows.
   \item Programming/scripting languages: R, Perl, Python, Bash, SQL, PHP and Matlab.
   \item Reproducible research tools: Git, Cloud computing (AWS and Digital Ocean), Markdown, and R Markdown.
\end{itemize}

%\section*{Hobbies and Interests}
%Sports (especially football), cycling, blogging, reading, and movies.

\section*{Academic References}

\begin{minipage}[ht]{.50\textwidth}
Dr. Reshmi G \\
Former Scientist at Rajiv Gandhi Centre for Biotechnology, \\
\href{https://www.researchgate.net/profile/Reshmi_Girijadevi}{https://www.researchgate.net/profile/Reshmi_Girijadevi}
Thiruvananthapural, Kerala 695014 \\
India \\
\href{mailto:reshmisuresh@gmail.com}{reshmisuresh@gmail.com} \\
Phone: +91 9947262407
\end{minipage}
\begin{minipage}[ht]{.50\textwidth}
Dr. Vinod Chandra S. S. \\
Director, Computer Centre, University of Kerala \\
/href{http://vinod.mirworks.in/home.php}{http://vinod.mirworks.in/home.php} \\
Thiruvananthapuram, Kerala 695034\\
India \\
\href{mailto:vinod@keralauniversity.ac.in}{vinod@keralauniversity.ac.in} \\
Phone: +81 45 503 9222
\end{minipage}

\section*{Publications}
% http://tex.stackexchange.com/questions/22645/hiding-the-title-of-the-bibliography
\begingroup
   \renewcommand{\section}[2]{}
   \bibliographystyle{unsrturl}
   \nocite{*}
   \bibliography{pub}
\endgroup

\vfill

\footnotesize
This C.V. was prepared in \LaTeX\ and is available at \href{https://github.com/viv3kanand/CV}{https://github.com/viv3kanand/CV}.

\end{document}
